\documentclass[a4paper,10pt]{article}
\usepackage[left=1in,right=1in,top=1in,bottom=1in]{geometry}
\usepackage{enumitem}
\usepackage{titlesec}
\usepackage{xcolor}
\usepackage{hyperref}
\usepackage{parskip}

\definecolor{myblue}{RGB}{0,76,153}
\hypersetup{
    colorlinks=true,
    linkcolor=myblue,
    urlcolor=myblue,
}

\pagestyle{empty}

\titleformat{\section}{
    \vspace{-4pt}\scshape\raggedright\large
}{}{0em}{}[\color{myblue}\titlerule\vspace{-5pt}]

\setlist[itemize]{left=0.5in}

\begin{document}

\begin{center}
    \textbf{\LARGE Sachin Shivaji Doddamani}
    
    Lakshmeshwar, Karnataka | 6361216081 | \href{mailto:ronhartforddelta19@gmail.com}{ronhartforddelta19@gmail.com} | \href{https://www.linkedin.com/in/sachin-doddamani-531a5a174/}{LinkedIn}
\end{center}

\section*{Professional Summary}
    Energetic computer science graduate with a strong machine learning and data analysis background accompanied by excellent communication and interpersonal skills. Proficient in deep learning and statistical techniques for data science. Extended knowledge of ETL operations, database management systems, and the FinTech ecosystem. Proven track record of successful internships and projects.

\section*{Education}
    \textbf{Jawahar Navodaya Vidyalaya, Dharwad }\\
    Class X | Dharwad, Karnataka\\
    CGPA: 8.6 (2014)
    
    \textbf{Shantiniketan PU College, Dharwad }\\
    Class XII | Dharwad, Karnataka\\
    Percentage: 64\% (2016)
    
    \textbf{VTU Belgaum }\\
    Bachelor of Engineering in Computer Science and Engineering | Chamarajanagar, Karnataka\\
    CGPA: 7.45 (2017-21)
    
    \textbf{National Institute of Technology, Karnataka }\\
    Master of Technology in Computer Science | Surathkal, Karnataka\\
    CGPA: 7.0 (2022-present)

\section*{Skills}
    \begin{itemize}
        \item Statistical Machine Learning
        \item Deep Learning with Keras and TensorFlow
        \item MongoDB and MySQL Proficiency
        \item FinTech Ecosystem 
        \item Stakeholder Relationship Management
    \end{itemize}

\section*{Experience}
    \textbf{Intern | Tech Fortune Technologies \hfill 07/2020 - 08/2020}\\
    Bangalore, Karnataka
    \begin{itemize}
        \item Learned core concepts of Data Science and feature engineering
        \item Designed and implemented data analysis projects based on publicly available datasets.
    \end{itemize}

\section*{Certifications}
    
    \textbf{MongoDB and Document Model} (MongoDB)
    \begin{itemize}
        \item Issued: Oct 2023
        \item Credential ID: MDB0zdaccp1ok
    \end{itemize}
    
    \textbf{Artificial Intelligence with Python} (Great Learning)
    \begin{itemize}
        \item Issued: Feb 2022
    \end{itemize}
    
    \textbf{SQL} (HackerRank)
    \begin{itemize}
        \item Issued: Sep 2021
        \item Credential ID: ec6e94bca304
    \end{itemize}
    
\section*{Projects}
    \textbf{Semi-automatic Depth-Map Generation of Images using Random Walk Segmentation in MATLAB (VTU)}\\
    \textbf{Objective:} Created a model that autonomously generates depth maps from images, with semi or least-human intervention. Implemented in MATLAB, utilizing the image processing toolbox alongside Gaussian mixture models and the random walk algorithm. 
    
    \textbf{Crop Production and Yield Prediction in Karnataka State with Machine Learning Algorithms (NITK)}\\
    \textbf{Objective:} Employed EDA and machine learning solutions to forecast crop yields in Karnataka by analyzing weather, soil quality, rainfall, and historical crop data. Implemented Random Forest and Gradient Boosting regression methods using Python's Scikit-learn and Pandas libraries. 
    
    \textbf{E-wallet and Mobile Payment Transaction Anomaly Detection to Prevent Money Laundering (Ongoing)}\\
    \textbf{Objective:} Developing a robust system integrating smart contracts and machine learning algorithms for real-time anomaly detection and enhancing user security in E-wallet and Mobile payments to combat money laundering.

\end{document}
